\selecthungarian
%--------------------------------------------------------------------------------------
% Rovid formai es tartalmi tajekoztato
%--------------------------------------------------------------------------------------

\footnotesize
\begin{center}
\large
\textbf{\Large Általános információk, a diplomaterv szerkezete}\\
\end{center}

A diplomaterv szerkezete a BME Villamosmérnöki és Informatikai Karán:
\begin{enumerate}
\item	Diplomaterv feladatkiírás
\item	Címoldal
\item	Tartalomjegyzék
\item	A diplomatervező nyilatkozata az önálló munkáról és az elektronikus adatok kezeléséről
\item	Tartalmi összefoglaló magyarul és angolul
\item	Bevezetés: a feladat értelmezése, a tervezés célja, a feladat indokoltsága, a diplomaterv felépítésének rövid összefoglalása
\item	A feladatkiírás pontosítása és részletes elemzése
\item	Előzmények (irodalomkutatás, hasonló alkotások), az ezekből levonható következtetések
\item	A tervezés részletes leírása, a döntési lehetőségek értékelése és a választott megoldások indoklása
\item	A megtervezett műszaki alkotás értékelése, kritikai elemzése, továbbfejlesztési lehetőségek
\item	Esetleges köszönetnyilvánítások
\item	Részletes és pontos irodalomjegyzék
\item	Függelék(ek)
\end{enumerate}

Felhasználható a következő oldaltól kezdődő \LaTeX diplomatervsablon dokumentum tartalma. 

A diplomaterv szabványos méretű A4-es lapokra kerüljön. Az oldalak tükörmargóval készüljenek (mindenhol 2,5~cm, baloldalon 1~cm-es kötéssel). Az alapértelmezett betűkészlet a 12 pontos Times New Roman, másfeles sorközzel, de ettől kismértékben el lehet térni, ill. más betűtípus használata is megengedett.

Minden oldalon -- az első négy szerkezeti elem kivételével -- szerepelnie kell az oldalszámnak.

A fejezeteket decimális beosztással kell ellátni. Az ábrákat a megfelelő helyre be kell illeszteni, fejezetenként decimális számmal és kifejező címmel kell ellátni. A fejezeteket decimális aláosztással számozzuk, maximálisan 3 aláosztás mélységben (pl. 2.3.4.1.). Az ábrákat, táblázatokat és képleteket célszerű fejezetenként külön számozni (pl. 2.4. ábra, 4.2. táblázat vagy képletnél (3.2)). A fejezetcímeket igazítsuk balra, a normál szövegnél viszont használjunk sorkiegyenlítést. Az ábrákat, táblázatokat és a hozzájuk tartozó címet igazítsuk középre. A cím a jelölt rész alatt helyezkedjen el.

A képeket lehetőleg rajzoló programmal készítsék el, az egyenleteket egyenlet-szerkesztő segítségével írják le (A \LaTeX~ehhez kézenfekvő megoldásokat nyújt).

Az irodalomjegyzék szövegközi hivatkozása történhet sorszámozva (ez a preferált megoldás) vagy a Harvard-rendszerben (a szerző és az évszám megadásával). A teljes lista névsor szerinti sorrendben a szöveg végén szerepeljen (sorszámozott irodalmi hivatkozások esetén hivatkozási sorrendben). A szakirodalmi források címeit azonban mindig az eredeti nyelven kell megadni, esetleg zárójelben a fordítással. A listában szereplő valamennyi publikációra hivatkozni kell a szövegben (a \LaTeX-sablon a Bib\TeX~segítségével mindezt automatikusan kezeli). Minden publikáció a szerzők után a következő adatok szerepelnek: folyóirat cikkeknél a pontos cím, a folyóirat címe, évfolyam, szám, oldalszám tól-ig. A folyóiratok címét csak akkor rövidítsük, ha azok nagyon közismertek vagy nagyon hosszúak. Internetes hivatkozások megadásakor fontos, hogy az elérési út előtt megadjuk az oldal tulajdonosát és tartalmát (mivel a link egy idő után akár elérhetetlenné is válhat), valamint az elérés időpontját.

\vspace{5mm}
Fontos:
\begin{itemize}
	\item A szakdolgozatkészítő / diplomatervező nyilatkozata (a jelen sablonban szereplő szövegtartalommal) kötelező előírás, Karunkon ennek hiányában a szakdolgozat/diplomaterv nem bírálható és nem védhető!
	\item Mind a dolgozat, mind a melléklet maximálisan 15~MB méretű lehet!
\end{itemize}

\vspace{5mm}
\begin{center}
Jó munkát, sikeres szakdolgozatkészítést, ill. diplomatervezést kívánunk!
\end{center}

\normalsize
\selectthesislanguage
