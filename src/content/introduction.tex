%----------------------------------------------------------------------------
\chapter{\bevezetes}
%----------------------------------------------------------------------------
%----------------------------------------------------------------------------
\section{Rendszer modellezés}
%----------------------------------------------------------------------------

Napjainkban a modellezés fogalom nagyon elterjedté vált az ipari világ valamennyi ágában. A rendszermodellezés egy olyan folyamat, amelyben egy rendszert írunk le különböző absztrakciós szinteken. Minden modell egy különálló nézőpontból mutatja be a rendszert.

A modellek több nézőpontból írják le a rendszert:
 \begin{itemize}
 	\item Külső perspektíva, ahol a rendszer kontextusát és környezetét írjuk le
 	\item Belső perspektíva, ahol a rendszer belső komponensei közti és a környezettel való kapcsolatot írjuk le
 	\item Strukturális perspektíva, ahol a rendszer felépítését és a feldolgozott adatok struktúráját ábrázoljuk
 	\item Viselkedési perspektíva, ahol a rendszer viselkedését és különböző eseményekre való reakcióját részletezzük
 \end{itemize}

Mindehhez valamilyen grafikus jelölésre van szükségünk, a legismertebb ábrázolási szemantika a Unified Modeling Language (UML). Az öt legelterjedtebb UML diagram:
 \begin{itemize}
	\item Működési (Activity) - folyamatokat és adatfeldolgozási lépéseket modellezhetünk
	\item Használati eset (Use case) - a rendszer és a környezete közti interakciókat írjuk le
	\item Szekvencia (Sequence) - aktorok és rendszer továbbá belső komponensek közti kommunikációt ábrázolják
	\item Osztály (Class) - objektumokat és ezek közti hierarchikus kapcsolatokat tudjuk leírni
	\item Állapot (State) - a rendszer viselkedését modellezhetjük különböző belső vagy külső események hatására
\end{itemize}

%----------------------------------------------------------------------------
\section{Gamma keretrendszer}
%----------------------------------------------------------------------------


