%----------------------------------------------------------------------------
\chapter{\bevezetes}
%----------------------------------------------------------------------------
%----------------------------------------------------------------------------
\section{Rendszer modellezés}
%----------------------------------------------------------------------------

Napjainkban a modellezés fogalom nagyon elterjedté vált az ipari világ valamennyi ágában. A rendszermodellezés egy olyan folyamat, amelyben egy rendszert írunk le különböző absztrakciós szinteken. Minden modell egy különálló nézőpontból mutatja be a rendszert.

A modellek több nézőpontból írják le a rendszert:
 \begin{itemize}
 	\item Külső perspektíva, ahol a rendszer kontextusát és környezetét írjuk le
 	\item Belső perspektíva, ahol a rendszer belső komponensei közti és a környezettel való kapcsolatot írjuk le
 	\item Strukturális perspektíva, ahol a rendszer felépítését és a feldolgozott adatok struktúráját ábrázoljuk
 	\item Viselkedési perspektíva, ahol a rendszer viselkedését és különböző eseményekre való reakcióját részletezzük
 \end{itemize}

Mindehhez valamilyen grafikus jelölésre van szükségünk, a legismertebb ábrázolási szemantika a Unified Modeling Language (UML). Az öt legelterjedtebb UML diagram:
 \begin{itemize}
	\item Működési (Activity) - folyamatokat és adatfeldolgozási lépéseket modellezhetünk
	\item Használati eset (Use case) - a rendszer és a környezete közti interakciókat írjuk le
	\item Szekvencia (Sequence) - aktorok és rendszer továbbá belső komponensek közti kommunikációt ábrázolják
	\item Osztály (Class) - objektumokat és ezek közti hierarchikus kapcsolatokat tudjuk leírni
	\item Állapot (State) - a rendszer viselkedését modellezhetjük különböző belső vagy külső események hatására
\end{itemize}

%----------------------------------------------------------------------------
\section{Gamma keretrendszer}
%----------------------------------------------------------------------------
A Gamma keretrendszer (Gamma Állapotgép Kompozíciós Keretrendszer) egy olyan eszköz amivel komponens alapú reaktív rendszereket tudunk modellezni, verifikálni vagy akár kódot generálni. A gamma az egyetem Méréstechnika és Információs Rendszerek tanszékén készült, fő fejlesztője Graics Bence.

Reaktív rendszernek nevezzük azt az architekturális stílust, amely segítségével lehetőségünk van különálló alkalmazások egységként való kezelésére úgy, hogy ezek a komponensek továbbá képesek egymás eseményeire és a környezetükre reagálni.

A keretrendszer Yakindu (Yakindu Statechart Tools) alapokon készült, ami egy nyílt forráskódú állapotgépeket modellező eszköz. A gamma ezt továbbviszi és egy magasabb modellezési réteget biztosít a felhasználók számára, amely segítségével hierarchikus állapotgép hálózatokat tudnak kialakítani. A Gamma képes egyszerű állapotgépek és komplex állapotgép hálózatok modell verifikációjára, mindehhez felhasználja az UPAAL-t ami egy modell ellenőrző eszköz. Továbbá, a Gamma segítségével lehetőségünk van a teljes rendszerhez kódot generálni, jelenleg a Java nyelv támogatott.

%----------------------------------------------------------------------------
\section{Feladat áttekintés}
%----------------------------------------------------------------------------
Ahogy a modellalapú fejlesztési módszerek egyre elterjedtebbé válnak, úgy erősödik az igény a korszerűbb eszköztámogatásra is. Ennek egyik eleme, hogy az olyan funkciók mint pl. a kódgenerálás, modelltranszformációk, vagy modellellenőrzés a szoftverfejlesztésnél megszokott módon egy folytonos integrációs folyamat részét képezzék. Ráadásul ezen funkciók egy része igen erőforrásigényes, így adja magát az automatikusan skálázódó, felhőalapú szolgáltatások megjelenése is.


A Gamma egy Eclipse-alapú eszköz, így jelen formájában nem felel meg a fentebb vázolt modern követelményeknek. A feladat ennek megfelelően az, hogy az alkalmazás szolgáltatásként is használható részeit parancssoron keresztül, illetve OpenAPI segítségével webes szolgáltatásként is elérhetővé tegyük.

%----------------------------------------------------------------------------
\section{Motiváció}
%----------------------------------------------------------------------------

Ebbe a fejezetben részletezni fogom a feladatban említett felhő és webes szolgáltatások előnyeit.

%----------------------------------------------------------------------------
\subsection{Felhő alapú megoldások}
%----------------------------------------------------------------------------



%----------------------------------------------------------------------------
\subsection{Webes szolgáltatások}
%----------------------------------------------------------------------------

