\pagenumbering{roman}
\setcounter{page}{1}

\selecthungarian

%----------------------------------------------------------------------------
% Abstract in Hungarian
%----------------------------------------------------------------------------
\chapter*{Kivonat}\addcontentsline{toc}{chapter}{Kivonat}
Napjainkban az informatikai világ exponenciális növekedésének hatására a modern alkalmazások komplexitása is jelentősen megnőtt, az ezzel járó pluszmunka a fejlesztők életét nehezíti. Erre megoldást szolgál a modellalapú szoftverfejlesztés paradigma, amely különböző vizuális modellek használatát írja elő. Az ilyen modellek segítségével minden fejlesztési ciklus átláthatóbbá és rövidebbé válik, legyen az tervezés, implementáció vagy dokumentáció.

A Gamma keretrendszer egy olyan eszköztár, ami a fentebb leírtakra illeszkedik. Pontosabban, egy olyan eszköz, amivel komponens alapú reaktív rendszereket tudunk modellezni, az így létrehozott modelleken verifikációt futtatni vagy akár kódot generálni. A Gamma jelenleg fejlesztői környezethez kötött, így nem tudja kihasználni az utóbbi években elterjedt felhő alapú szolgáltatások előnyeit.

A dolgozat a továbbfejlesztési lehetőséghez tartozó rendszert mutatja be. A feladat tehát ennek megfelelően az, hogy a Gammát mint kötött szoftver komponens felhőalapú szolgáltatássá alakítsuk át.

Az átalakított Gamma köré egy olyan rendszert alakítunk ki, amely képes kihasználni a felhőalapú megoldások mögött álló előnyöket. A dolgozat bemutatja, milyen technológiák alapján lett megvalósítva a rendszer, továbbá felvázol egy követelményhalmazt, ami egyértelműen meghatározza az alkalmazás funkcionális és nemfunkcionális aspektusait. Áttekinti az alkalmazás belső felépítését, a különböző komponensek feladatait és szekvenciadiagramok segítségével a rendszer viselkedésébe is betekintést nyújt. Zárásként bemutatja egy valós teszthalmazon, hogy pontosan hogyan működik a rendszer.

\vfill
\selectenglish


%----------------------------------------------------------------------------
% Abstract in English
%----------------------------------------------------------------------------
\chapter*{Abstract}\addcontentsline{toc}{chapter}{Abstract}

Nowadays, as a result of the exponential growth of the IT world, the complexity of modern applications has also increased significantly, and the extra work involved makes life difficult for developers. A solution to this is the model-based software development paradigm, which allows the use of different visual models. With such models, every development cycle becomes more transparent and shorter, be it design, implementation or documentation.


The Gamma Framework is a toolset that fits the ones described above. More specifically, it is a tool with which we can model component-based reactive systems, run verification on the models thus created, or even generate code. Gamma is currently tied to a development environment, so it cannot take advantage of the cloud-based services that have become widespread in recent years.


The dissertation presents a system that solves the issue described above. Accordingly, the task is to transform the Gamma Framework as a tied software component into a cloud-based service.

We are developing a system around the redesigned Gamma that can take advantage of the benefits behind cloud-based solutions. The dissertation presents the technologies on the basis of which the system was implemented, and outlines a set of requirements that clearly define the functional and non-functional aspects of the application. It provides an overview of the internal structure of the application, the tasks of the various components, and also provides insight into the behavior of the system using sequence diagrams. In the end, it shows you on a real test set exactly how the system works.
\vfill
\cleardoublepage

\selectthesislanguage

\newcounter{romanPage}
\setcounter{romanPage}{\value{page}}
\stepcounter{romanPage}
