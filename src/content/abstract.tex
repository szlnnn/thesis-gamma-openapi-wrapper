\pagenumbering{roman}
\setcounter{page}{1}

\selecthungarian

%----------------------------------------------------------------------------
% Abstract in Hungarian
%----------------------------------------------------------------------------
\chapter*{Kivonat}\addcontentsline{toc}{chapter}{Kivonat}
Napjainkban az informatikai világ exponenciális növekedésének hatására a modern alkalmazások komplexitása is jelentősen megnő, az ezzel járó pluszmunka a fejlesztők munkáját nehezíti. Erre megoldást szolgál a modellalapú szoftverfejlesztés paradigma, amely különböző vizuális modellek használatát írja elő. Az ilyen modellek segítségével minden fejlesztési ciklus átláthatóbbá és rövidebbé válik, legyen az tervezés, implementáció vagy dokumentáció.

A Gamma keretrendszer egy olyan eszköztár, ami a fentebb leírtakra illeszkedik. Pontosabban, egy olyan eszköz, amivel komponens alapú reaktív rendszereket tudunk modellezni, az így létrehozott modelleken verifikációt futtatni vagy akár kódot generálni. A Gamma egyik legnagyobb hiányossága, hogy fejlesztői környezethez kötött így nem tudja kihasználni az utóbbi években elterjedt felhő alapú szolgáltatások előnyeit.

A dolgozat erre a hiányosságra adott megoldást mutatja be. A feladat tehát ennek megfelelően az, hogy a gammát mint kötött szoftver komponens felhőalapú szolgáltatássá alakítsuk át.

Az átalakított gamma köré egy olyan rendszert alakítunk ki, amely képes kihasználni a felhőalapú megoldások mögött álló előnyöket. A dolgozat bemutatja, milyen technológiák alapján lett megvalósítva a rendszer, továbbá felvázol egy követelményhalmazt, ami egyértelműen meghatározza az alkalmazás funkcionális és nem funkcionális aspektusát is. Áttekinti az alkalmazás belső felépítését, a különböző komponensek feladatait és szekvencia diagramok segítségével a rendszer viselkedésébe is betekintést nyújt. Zárásként bemutatja egy valós teszthalmazon, hogy pontosan hogyan működik a rendszer.


\vfill
\selectenglish


%----------------------------------------------------------------------------
% Abstract in English
%----------------------------------------------------------------------------
\chapter*{Abstract}\addcontentsline{toc}{chapter}{Abstract}

This document is a \LaTeX-based skeleton for BSc/MSc~theses of students at the Electrical Engineering and Informatics Faculty, Budapest University of Technology and Economics. The usage of this skeleton is optional. It has been tested with the \emph{TeXLive} \TeX~implementation, and it requires the PDF-\LaTeX~compiler.


\vfill
\cleardoublepage

\selectthesislanguage

\newcounter{romanPage}
\setcounter{romanPage}{\value{page}}
\stepcounter{romanPage}