\pagenumbering{roman}
\setcounter{page}{1}

\selecthungarian

%----------------------------------------------------------------------------
% Abstract in Hungarian
%----------------------------------------------------------------------------
\chapter*{Kivonat}\addcontentsline{toc}{chapter}{Kivonat}
Napjainkban az informatikai világ exponenciális növekedésének hatására a modern alkalmazások komplexitása is jelentősen megnőtt, az ezzel járó pluszmunka a fejlesztők életét nehezíti. Erre megoldást szolgál a modellalapú szoftverfejlesztés paradigma, amely különböző vizuális modellek használatát írja elő. Az ilyen modellek segítségével minden fejlesztési ciklus átláthatóbbá és rövidebbé válik, legyen az tervezés, implementáció vagy dokumentáció.

A Gamma keretrendszer egy olyan eszköztár, ami a fentebb leírtakra illeszkedik. Pontosabban, egy olyan eszköz, amivel komponens alapú reaktív rendszereket tudunk modellezni, az így létrehozott modelleken verifikációt futtatni vagy akár kódot generálni. A Gamma jelenleg fejlesztői környezethez kötött, így nem tudja kihasználni az utóbbi években elterjedt felhő alapú szolgáltatások előnyeit.

A dolgozat a továbbfejlesztési lehetőséghez tartozó rendszert mutatja be. A feladat tehát ennek megfelelően az, hogy a Gammát mint kötött szoftver komponens felhőalapú szolgáltatássá alakítsuk át.

Az átalakított Gamma köré egy olyan rendszert alakítunk ki, amely képes kihasználni a felhőalapú megoldások mögött álló előnyöket. A dolgozat bemutatja, milyen technológiák alapján lett megvalósítva a rendszer, továbbá felvázol egy követelményhalmazt, ami egyértelműen meghatározza az alkalmazás funkcionális és nemfunkcionális aspektusait. Áttekinti az alkalmazás belső felépítését, a különböző komponensek feladatait és szekvenciadiagramok segítségével a rendszer viselkedésébe is betekintést nyújt. Zárásként bemutatja egy valós teszthalmazon, hogy pontosan hogyan működik a rendszer.
